\documentclass[11pt,a4paper,italian]{article}

\usepackage[italian]{babel}
\usepackage[T1]{fontenc}
\usepackage{geometry}	% Set document margins
\usepackage{amsmath}
\usepackage{amssymb}
\usepackage{mathtools}
\usepackage[hidelinks]{hyperref}
\usepackage{color}
\usepackage{graphicx}
\usepackage{textcomp}	% For single quotes in codes
\usepackage{indentfirst}	% For indenting first paragraph
\usepackage{pgfplots}
\usepackage{forest}
\usetikzlibrary{arrows}
\newcommand{\NA}{-}
\usepackage{ascii}

\title{\textbf{Laboratorio di Algoritmi e Strutture Dati 2020/2021} --- Seconda parte}
\date{\today}
\author{\textbf{Mattia Bonaccorsi} --- 124610 -- \href{mailto:bonaccorsi.mattia@spes.uniud.it}{bonaccorsi.mattia@spes.uniud.it}
   \and \textbf{Muhamed Kouate} --- 137359 -- \href{mailto:kouate.muhamed@spes.uniud.it}{kouate.muhamed@spes.uniud.it}
   \and \textbf{Enrico Stefanel} --- 137411 -- \href{mailto:stefanel.enrico@spes.uniud.it}{stefanel.enrico@spes.uniud.it}
   \and \textbf{Andriy Torchanyn} --- 139535 -- \href{mailto:torchanyn.andriy@spes.uniud.it}{torchanyn.andriy@spes.uniud.it}
   }


\usepackage{listings} 
\usepackage{setspace} 


\definecolor{halfgray}{gray}{0.55}
\definecolor{snow}{RGB}{250, 250, 250}
\definecolor{green}{RGB}{0, 128, 0}
\definecolor{cyan}{RGB}{64, 128, 128}
\definecolor{folly}{rgb}{1.0, 0.0, 0.31}

\lstdefinelanguage{python}{
	morekeywords={and,assert,break,class,def,elif,else,for,from,global,if,import,as,in,is,lambda,not,or,pass,print,return,try,while},
	morekeywords=[2]{abs,all,any,bool,chr,dict,dir,divmod,enumerate,eval,file,filter,float,format,globals,help,id,
	input,int,len,list,locals,long,map,min,next,pow,range,input,round,set,str,sum,super,tuple,type,unichr,unicode,vars}, keywordstyle=\color{green},
	morestring=[b]", morestring=[b]', stringstyle=\color{folly}, showstringspaces=false, upquote=true,
    morecomment=[l]\#, commentstyle=\color{cyan},
    frame=single, framexleftmargin=2em, backgroundcolor=\color{snow},
    numbers=left, numberstyle=\scriptsize\color{halfgray}\ttfamily, xleftmargin=1.5em,
    inputpath=sources/,
    title=\footnotesize\ttfamily\lstname, captionpos=b,
    basicstyle=\ttfamily\small
}

\begin{document}

  \tikzset{%
    /forest,
    forest node/.style={circle split,draw, text width=1.75em, text centered,s sep+=1cm,l sep-=.5cm, font=\footnotesize},
%    forest node/.style={rectangle split,rectangle split parts=2,draw, text width=3em, text centered, l sep+=.25cm,s sep+=.25cm},
    b/.append style={draw=black, very thick},
    r/.append style={draw=red, very thick},
    nill/.append style={rectangle,draw=black, very thick, text width=.05em,l sep-=3cm},
    null/.append style={rectangle, text width=.05em,l sep-=3cm}
  }



\maketitle
\tableofcontents


\clearpage

\section{Alberi binari di ricerca semplici}

\subsection{Definizione di \textit{BST}}  
Un \textit{albero binario di ricerca} (o BST) $T$ è una struttura dati ad albero, in cui valgono le seguenti proprietà:
\begin{equation}
\begin{split}
	\forall x \in T,\; \forall y \in left(T) \rightarrow y.key < x.key	 \\
	\forall x \in T,\; \forall z \in right(T) \rightarrow z.key > x.key
\end{split}  \tag{$\star$}
\end{equation}
dove $k.key$ indica il valore della chiave di $k$, e $left(B)$ (rispettivamente $right(B)$) indica il sotto-albero sinistro (rispettivamente destro) di $B$.
\\

\textbf{Esempio} Un \textit{BST} di tipo semplice, in cui ogni nodo contiene una chiave numerica dell'insieme $\{1,2,3,4,5,6,10\}$ e un campo alfanumerico di tipo stringa, è il seguente:
\begin{center}
\begin{forest}
for tree={forest node}
	[6\nodepart{lower}six
  		[5\nodepart{lower}five
    		[1\nodepart{lower}one
				[,null]
				[2\nodepart{lower}two
					[,null]
					[4\nodepart{lower}four
						[3\nodepart{lower}three
							[,null]
							[,null]
						]
						[,null]
					]
				]
			]
    		[,null]
  		]
  		[10\nodepart{lower}ten
			[,null]
			[,null]
		]
	]
\end{forest}
\end{center}

Bisogna notare che non è l'unico \textit{BST} costruibile partendo dallo stesso insieme di chiavi. Un'alternativa, per esempio, potrebbe essere stata quella di utilizzare il valore minore come chiave per la radice dell'albero, e attaccare in ordine crescente le altre chiavi, ognuna come figlio destro del nodo precedente.

\subsection{Implementazione della struttura dati}
Per implementare la struttura dati dell'Albero binario di ricerca semplice, abbiamo innanzitutto bisogno di definire una classe \texttt{Node} per le istanze dei Nodi che compongono il BST:
\lstinputlisting[language=Python,linerange={1-7}]{bst.py} 

Una volta definita la classe \texttt{Node}, possiamo procedere con l'implementazione dell'inserimento di un Nodo nel BST:
\lstinputlisting[language=Python,linerange={10-28}]{bst.py}

Definiamo poi una procedura, anche questa ricorsiva, per la ricerca di un Nodo all'interno di un Albero:
\lstinputlisting[language=Python,linerange={64-80}]{bst.py}

\subsubsection{Osservazioni sull'implementazione della struttura dati}
Le procedure per l'inserimento e la ricerca di un nodo all'interno di un BST sono state scritte in maniera ricorsiva, per chiarezza. Essendo però una \textit{ricorsione di coda}, è immediato trasformare le funzioni per ottenere delle funzioni iterative.

La funzione per l'inserimento, scritta in maniera iterativa, sarebbe la seguente:
\lstinputlisting[language=Python,linerange={30-61}]{bst.py}
, mentre la funzione di ricerca sarebbe scritta in questo modo:
\lstinputlisting[language=Python,linerange={82-98}]{bst.py}



\section{Alberi binari di ricerca di tipo AVL}

\subsection{Definizione di Albero \textit{AVL}}  
Un \textit{albero AVL} $T$ è un \textit{BST} ($\star$), in cui vale la seguente proprietà:
\begin{equation}
	\forall x \in T \rightarrow | h(left(x)) - h(right(x)) | \leq 1	 \tag{$\ast$}
\end{equation}
dove $h(k)$ indica il valore dell'altezza dell'albero radicato in $k$, e $left(B)$ (rispettivamente $right(B)$) indica il sotto-albero sinistro (rispettivamente destro) di $B$.
\\

\textbf{Esempio} Un Albero \textit{AVL} in cui ogni nodo contiene una chiave numerica dell'insieme $\{1,2,3,4,5,6,9,10\}$ e un campo alfanumerico di tipo stringa, è il seguente:
\begin{center}
\begin{forest}
for tree={forest node}
	[4\nodepart{lower}four
  		[2\nodepart{lower}two
    		[1\nodepart{lower}one
				[,null]
				[,null]
			]
    		[3\nodepart{lower}three
				[,null]
				[,null]
			]
  		]
  		[6\nodepart{lower}six
    		[5\nodepart{lower}five
				[,null]
				[,null]
			]
    		[10\nodepart{lower}ten
				[9\nodepart{lower}nine
					[,null]
					[,null]
				]
				[,null]
			]
  		]
	]
\end{forest}
\end{center},
dove, ad esempio, $left(\texttt{root})$ ha altezza 2, mentre $right(\texttt{root})$ ha altezza 3.

%\subsection{Implementazione della struttura dati}  
%\lstinputlisting[language=Python]{avl.py} 

\clearpage

\section{Alberi binari di ricerca di tipo Red-Black}

\subsection{Definizione di \textit{RB Tree}}  
Un \textit{albero di tipo Red-Black} (o \textit{RB Tree}) $T$ è un \textit{BST} ($\star$), in cui ogni nodo ha associato un campo "colore", che può assumere valore \textit{rosso} o \textit{nero}, ed inoltre vale che:
\begin{equation}
		\forall x \in T \rightarrow h_b(left(x)) = h_b(right(x))	 \tag{$\bullet$}
\end{equation}
dove $h_b(x)$ indica l'altezza nera dell'albero radicato in $x$, ovvero il massimo numero di nodi neri lungo un possibile cammino da $x$ a una foglia.
\\

\textbf{Esempio} Un \textit{BST} di tipo Red-Black, in cui ogni nodo contiene una chiave numerica dell'insieme $\{4,5,6,7,8,9,10,11\}$ e un campo alfanumerico di tipo stringa, è il seguente:
\begin{center}
\begin{forest}
for tree={forest node}
      [8\nodepart{lower}eight, b
        [6\nodepart{lower}six, r
          [4\nodepart{lower}four, b
          	[, nill]
          	[5\nodepart{lower}five, r
				[, nill]
				[, nill]
			]
          ]
          [7\nodepart{lower}seven, b
            [, nill]
			[, nill]
          ]
        ]
        [10\nodepart{lower}ten, b,
          [9\nodepart{lower}nine, r
          	[, nill]
			[, nill]
          ]
          [11\nodepart{lower}eleven, r
          	[, nill]
			[, nill]
          ]
        ]
      ]
 \end{forest}
\end{center}

%\subsection{Implementazione della struttura dati}  
%\lstinputlisting[language=Python]{rbt.py} 

\clearpage

\section{Calcolo della complessità}
Implementate le tre strutture dati precedentemente descritte utilizzando il linguaggio Python, si è poi proceduto a calcolare i tempi medi per la ricerca e l'inserimento di $n$ chiavi generate in modo pseudo-casuale.

\subsection{Caso random}
Lorem ipsum dolor sit amet, consectetur adipiscing elit. Mauris dignissim vitae diam et placerat. Nulla egestas dui id ultrices hendrerit. Morbi in mi sed velit sollicitudin dictum ac vitae justo. Nunc auctor aliquam lectus nec aliquet. Aenean gravida elit sit amet pellentesque malesuada. Sed vel dolor porta, vestibulum nulla non, rutrum dolor. Morbi pretium mollis urna, at tempus erat aliquam ut. Ut id pellentesque elit. Etiam at dolor eget dui vehicula tempor. Nam mattis nec massa ut volutpat. Morbi id arcu sit amet augue faucibus eleifend in a metus.

\begin{center}
\begin{minipage}{\textwidth}
\begin{tikzpicture}[scale=0.39]
\pgfplotsset{
    width=1.25\textwidth,
    height=1\textwidth
}
\begin{axis}[
	title={Tempi di ricerca e inserimento per alberi binari di ricerca},
    axis line style={->},
    xlabel={\small Dimensione dell'input (\textit{numero di nodi})},
    ylabel={\small Tempo (\textit{sec})},
	grid,
	grid style={dotted,gray},
	axis x line=bottom,
    axis y line=left,
    legend style={ at={(0.95,0.05)}, anchor=south east}
    ]
\addplot[mark=*, blue] table [x=size, y=amortizedTimes_BST, col sep=comma]{sources/results-random.csv};
\addplot[mark=*, orange] table [x=size, y=amortizedTimes_AVL, col sep=comma]{sources/results-random.csv};
\addplot[mark=*, green] table [x=size, y=amortizedTimes_RBT, col sep=comma]{sources/results-random.csv};
\addlegendentry{Alberi BST semplici}
\addlegendentry{Alberi BST di tipo \textit{AVL}}
\addlegendentry{Alberi BST di tipo \textit{RB}}
\end{axis}
\end{tikzpicture}
\begin{tikzpicture}[scale=0.39]
\pgfplotsset{
    width=1.25\textwidth,
    height=1\textwidth
}
\begin{axis}[
	title={Tempi di ricerca e inserimento per alberi binari di ricerca},
	xmode=log,
	ymode=log,
    axis line style={->},
    xlabel={\small Dimensione dell'input (\textit{numero di nodi})},
    ylabel={\small Tempo (\textit{sec})},
	grid,
	grid style={dotted,gray},
	axis x line=bottom,
    axis y line=left,
    legend style={ at={(0.95,0.05)}, anchor=south east}
	]
\addplot[mark=*, blue] table [x=size, y=amortizedTimes_BST, col sep=comma]{sources/results-random.csv};
\addplot[mark=*, orange] table [x=size, y=amortizedTimes_AVL, col sep=comma]{sources/results-random.csv};
\addplot[mark=*, green] table [x=size, y=amortizedTimes_RBT, col sep=comma]{sources/results-random.csv};
\addlegendentry{Alberi BST semplici}
\addlegendentry{Alberi BST di tipo \textit{AVL}}
\addlegendentry{Alberi BST di tipo \textit{RB}}
\end{axis}
\end{tikzpicture}
\end{minipage}
\end{center}
Nullam eu nisl eget neque lacinia pulvinar. Maecenas sit amet aliquam metus. Etiam volutpat quam a mi varius, malesuada aliquet augue porttitor. Aenean ut imperdiet libero. In hac habitasse platea dictumst. Nulla ultrices consequat neque, vel dictum sem fringilla at. Phasellus in sapien sit amet lectus rutrum vestibulum id vitae neque. Suspendisse vitae felis vitae velit hendrerit mollis.


\subsection{Caso sorted}
Lorem ipsum dolor sit amet, consectetur adipiscing elit. Mauris dignissim vitae diam et placerat. Nulla egestas dui id ultrices hendrerit. Morbi in mi sed velit sollicitudin dictum ac vitae justo. Nunc auctor aliquam lectus nec aliquet. Aenean gravida elit sit amet pellentesque malesuada. Sed vel dolor porta, vestibulum nulla non, rutrum dolor. Morbi pretium mollis urna, at tempus erat aliquam ut. Ut id pellentesque elit. Etiam at dolor eget dui vehicula tempor. Nam mattis nec massa ut volutpat. Morbi id arcu sit amet augue faucibus eleifend in a metus.

\begin{center}
\begin{minipage}{\textwidth}
\begin{tikzpicture}[scale=0.39]
\pgfplotsset{
    width=1.25\textwidth,
    height=0.95\textwidth
}
\begin{axis}[
	title={Tempi di ricerca e inserimento per alberi binari di ricerca},
    axis line style={->},
    xlabel={\small Dimensione dell'input (\textit{numero di nodi})},
    ylabel={\small Tempo (\textit{sec})},
	grid,
	grid style={dotted,gray},
	axis x line=bottom,
    axis y line=left,
    legend style={ at={(0.95,0.05)}, anchor=south east}
    ]
\addplot[mark=*, blue] table [x=size, y=amortizedTimes_BST, col sep=comma]{sources/results-sorted.csv};
\addplot[mark=*, orange] table [x=size, y=amortizedTimes_AVL, col sep=comma]{sources/results-sorted.csv};
\addplot[mark=*, green] table [x=size, y=amortizedTimes_RBT, col sep=comma]{sources/results-sorted.csv};
\addlegendentry{Alberi BST semplici}
\addlegendentry{Alberi BST di tipo \textit{AVL}}
\addlegendentry{Alberi BST di tipo \textit{RB}}
\end{axis}
\end{tikzpicture}
\begin{tikzpicture}[scale=0.39]
\pgfplotsset{
    width=1.25\textwidth,
    height=0.95\textwidth
}
\begin{axis}[
	title={Tempi di ricerca e inserimento per alberi binari di ricerca},
	xmode=log,
	ymode=log,
    axis line style={->},
    xlabel={\small Dimensione dell'input (\textit{numero di nodi})},
    ylabel={\small Tempo (\textit{sec})},
	grid,
	grid style={dotted,gray},
	axis x line=bottom,
    axis y line=left,
    legend style={ at={(0.95,0.05)}, anchor=south east}
	]
\addplot[mark=*, blue] table [x=size, y=amortizedTimes_BST, col sep=comma]{sources/results-sorted.csv};
\addplot[mark=*, orange] table [x=size, y=amortizedTimes_AVL, col sep=comma]{sources/results-sorted.csv};
\addplot[mark=*, green] table [x=size, y=amortizedTimes_RBT, col sep=comma]{sources/results-sorted.csv};
\addlegendentry{Alberi BST semplici}
\addlegendentry{Alberi BST di tipo \textit{AVL}}
\addlegendentry{Alberi BST di tipo \textit{RB}}
\end{axis}
\end{tikzpicture}
\end{minipage}
\end{center}
Nullam eu nisl eget neque lacinia pulvinar. Maecenas sit amet aliquam metus. Etiam volutpat quam a mi varius, malesuada aliquet augue porttitor. Aenean ut imperdiet libero. In hac habitasse platea dictumst. Nulla ultrices consequat neque, vel dictum sem fringilla at. Phasellus in sapien sit amet lectus rutrum vestibulum id vitae neque. Suspendisse vitae felis vitae velit hendrerit mollis.


\subsection{Caso smart}
Lorem ipsum dolor sit amet, consectetur adipiscing elit. Mauris dignissim vitae diam et placerat. Nulla egestas dui id ultrices hendrerit. Morbi in mi sed velit sollicitudin dictum ac vitae justo. Nunc auctor aliquam lectus nec aliquet. Aenean gravida elit sit amet pellentesque malesuada. Sed vel dolor porta, vestibulum nulla non, rutrum dolor. Morbi pretium mollis urna, at tempus erat aliquam ut. Ut id pellentesque elit. Etiam at dolor eget dui vehicula tempor. Nam mattis nec massa ut volutpat. Morbi id arcu sit amet augue faucibus eleifend in a metus.

\begin{center}
\begin{minipage}{\textwidth}
\begin{tikzpicture}[scale=0.39]
\pgfplotsset{
    width=1.25\textwidth,
    height=0.95\textwidth
}
\begin{axis}[
	title={Tempi di ricerca e inserimento per alberi binari di ricerca},
    axis line style={->},
    xlabel={\small Dimensione dell'input (\textit{numero di nodi})},
    ylabel={\small Tempo (\textit{sec})},
	grid,
	grid style={dotted,gray},
	axis x line=bottom,
    axis y line=left,
    legend style={ at={(0.95,0.05)}, anchor=south east}
    ]
\addplot[mark=*, blue] table [x=size, y=amortizedTimes_BST, col sep=comma]{sources/results-smart.csv};
\addplot[mark=*, orange] table [x=size, y=amortizedTimes_AVL, col sep=comma]{sources/results-smart.csv};
\addplot[mark=*, green] table [x=size, y=amortizedTimes_RBT, col sep=comma]{sources/results-smart.csv};
\addlegendentry{Alberi BST semplici}
\addlegendentry{Alberi BST di tipo \textit{AVL}}
\addlegendentry{Alberi BST di tipo \textit{RB}}
\end{axis}
\end{tikzpicture}
\begin{tikzpicture}[scale=0.39]
\pgfplotsset{
    width=1.25\textwidth,
    height=0.95\textwidth
}
\begin{axis}[
	title={Tempi di ricerca e inserimento per alberi binari di ricerca},
	xmode=log,
	ymode=log,
    axis line style={->},
    xlabel={\small Dimensione dell'input (\textit{numero di nodi})},
    ylabel={\small Tempo (\textit{sec})},
	grid,
	grid style={dotted,gray},
	axis x line=bottom,
    axis y line=left,
    legend style={ at={(0.95,0.05)}, anchor=south east}
	]
\addplot[mark=*, blue] table [x=size, y=amortizedTimes_BST, col sep=comma]{sources/results-smart.csv};
\addplot[mark=*, orange] table [x=size, y=amortizedTimes_AVL, col sep=comma]{sources/results-smart.csv};
\addplot[mark=*, green] table [x=size, y=amortizedTimes_RBT, col sep=comma]{sources/results-smart.csv};
\addlegendentry{Alberi BST semplici}
\addlegendentry{Alberi BST di tipo \textit{AVL}}
\addlegendentry{Alberi BST di tipo \textit{RB}}
\end{axis}
\end{tikzpicture}
\end{minipage}
\end{center}
Nullam eu nisl eget neque lacinia pulvinar. Maecenas sit amet aliquam metus. Etiam volutpat quam a mi varius, malesuada aliquet augue porttitor. Aenean ut imperdiet libero. In hac habitasse platea dictumst. Nulla ultrices consequat neque, vel dictum sem fringilla at. Phasellus in sapien sit amet lectus rutrum vestibulum id vitae neque. Suspendisse vitae felis vitae velit hendrerit mollis.


\end{document}
